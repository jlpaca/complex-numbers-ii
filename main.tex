\documentclass[tikz]{standalone}
\usepackage{amsmath}
\usepackage{amssymb}
\usepackage{mathtools}
\usepackage{contour}
\usepackage{kf-complex}
\usepackage{graphicx}

\begin{document}

\pgfmathsetmacro\dt{0.5}
% debug:
\kfsetfps{8}

%% 0.1 PARABLE OF THE CRAB
\newcommand{\crab}[4]{
  \pgfkeys{/pgf/fpu=false}
  \node[transform shape, rotate=(#4), scale={#1}] at ({#2}, {#3}) {\includegraphics[width=1cm]{resources/crab.png}};
  \pgfkeys{/pgf/fpu=true}

}
\newcommand\altfps{0.1}

%% \kfframes{3}{
  \pgfmathsetmacro\t{floor(\t/\altfps)*\altfps}
  \pgfmathsetmacro\vth{10}
  \pgfmathsetmacro\vx{8*cos(\vth)}
  \pgfmathsetmacro\vy{8*sin(\vth)}
  \pgfmathsetmacro\vox{-1}
  \pgfmathsetmacro\voy{2}
  
   % fill in background
   \fill[fg-a] (0, 0) rectangle (2*\halfwidth, 2*\halfheight);
   \color{white!10!black}

   \crab{1}{\vox+\vx*\t}{\voy+\vy*\t}{\vth+10*sin(10*\t r)}
   \draw[dashed, bg-b] (\vox, \voy-0.5) -- ({\vox+\vx*(\t+0.1)}, {\voy+\vy*(\t+0.1)-0.5});
}
          % 00:00
%% \kfframes{9}{ \pgfmathsetmacro\t{floor(\t/\altfps)*\altfps}

\newcommand\ta{0.5} % flag in
\newcommand\tb{2} % begin rotation to face flag
\newcommand\tc{3} % line in
\newcommand\td{5} % walk around

% fill in background
\fill[fg-a] (0, 0) rectangle (2*\halfwidth, 2*\halfheight); \color{white!10!black}

% flag
\begin{scope}[shift={(\halfwidth, \halfheight)}]

\pgfmathsetmacro\cth{-20} \pgfmathsetmacro\ctd{3.2}
\pgfmathsetmacro\cto{1}

% extend line
\kfclip{\ta}{\tend+1}{


%draw the flag
\draw[thick] (0, 0) -- (0, 1);
\fill (0, 0.7) rectangle (0.4, 1);

\kftween[in=\td, duration=(6*\dt)]{0}{180}
\pgfmathsetmacro\cth{\pgfmathresult+\cth}

\kftween[in=\tb, duration=\dt]{0}{90+\cth}
\pgfmathsetmacro\cthc\pgfmathresult

\crab{1}{\ctd*cos(\cth)}{\ctd*sin(\cth)}{\cthc}

\kftween[in=\tc, duration=\dt]{\ctd-\cto-0.1}{\cto/3}
\pgfmathsetmacro\ls\pgfmathresult

\kfclip{\tc}{\tend+1}{ \draw[dashed, ->, >=stealth]
({(\ctd-\cto)*cos(\cth)}, {(\ctd-\cto)*sin(\cth)}) --
({\ls*cos(\cth)}, {\ls*sin(\cth)}); } }
\end{scope}
}
             % 00:04
%% \kfframes{9}{

% fill in background
\fill[fg-a] (0, 0) rectangle (2*\halfwidth, 2*\halfheight);
\color{white!10!black}

\newcommand\taa{0.5}
\newcommand\ta{3}
\newcommand\tdelay{0.4}
\newcommand\tb{5}

\pgfmathsetmacro\cy{-0.5}
\pgfmathsetmacro\cx{1.5}

\kfclip{\taa}{\tend+1}{
\kftext[polar=false]{4.5}{\halfheight+\cy}{Crab Anatomy:}

\begin{scope}[shift={(\halfwidth+\cx, \halfheight+\cy)}]

\pgfmathsetmacro\al{2}

\crab{2}{0}{0}{0}

\kfclip{\ta}{\tend+1}{
\draw[->, >=stealth] (2, 0) -- ({2+\al}, 0);
}
\kfclip{\ta+\tdelay}{\tend+1}{
\node at (2+\al/2, 1) {\scriptsize direction};
\node at (2+\al/2, 0.5) {\scriptsize of travel};
}

\kfclip{\tb}{\tend+1}{
\draw[->, >=stealth] (0, 2) -- (0, 2 + \al);
}
\kfclip{\tb+\tdelay}{\tend+1}{
\node at (-1, 2+\al/2) {\rotatebox{90}{\scriptsize line}};
\node at (-0.5, 2+\al/2) {\rotatebox{90}{\scriptsize of sight}};
}

\end{scope}
}
}
    % 00:13
%% \kfframes{1.5}{
  % fill in background
  \fill[fg-a] (0, 0) rectangle (2*\halfwidth, 2*\halfheight);
  \color{white!10!black}
  
  \kftext[polar=false]{\halfwidth}{\halfheight}{Questions:}
}
               % 00:21+12
%% \kfframes{7}{
\pgfmathsetmacro\t{floor(\t/\altfps)*\altfps}

\newcommand\ta{0} % start moving
\newcommand\tb{3.5} % start switching
\newcommand\tdelay{0.75}

% fill in background
\fill[fg-a] (0, 0) rectangle (2*\halfwidth, 2*\halfheight); \color{white!10!black}

% flag
\begin{scope}[shift={(\halfwidth, \halfheight)}]

%    \pgfmathsetmacro\ctho{140}
\pgfmathsetmacro\ctd{3.2}
\pgfmathsetmacro\ctoa{5}
\pgfmathsetmacro\cto{1}
\pgfmathsetmacro\cte{\ctd+0.5}

%draw the flag
\draw[thick] (0, 0) -- (0, 1); \fill (0, 0.7) rectangle (0.4, 1);

\kftween[in=\ta, duration=(6*\dt)]{160}{340}
\pgfmathsetmacro\cth{\pgfmathresult}

\kfclip{0}{\tb}{
\pgfkeys{/pgf/fpu=false}
\draw[dashed, bg-b] ({\cte*cos(-20+\ctoa)}, {\cte*sin(-20+\ctoa)}) arc (-20+\ctoa:\cth+\ctoa:\cte);
\pgfkeys{/pgf/fpu=true}
}

\kfclip{\tb}{\tb+\tdelay}{
% ellipse
\pgfkeys{/pgf/fpu=false}
\begin{scope}[rotate=25]
\draw[bg-b, dashed, domain=-pi:pi, samples=80, variable=\tt]
plot ({1.4*\cte*cos(\tt r)}, {0.9*\cte*sin(\tt r)});
\end{scope}
\pgfkeys{/pgf/fpu=true}
}

\kfclip{\tb+\tdelay}{\tb+2*\tdelay}{
% wobblu
\draw [bg-b, dashed, smooth, domain=-pi:pi, samples=80, variable=\tt] 
plot ({(\ctd+0.3-0.4*cos(7*\tt r))*cos(\tt r)}, {(\ctd+0.3-0.4*cos(7*\tt r))*sin(\tt r)});
}

\kfclip{\tb+2*\tdelay}{\tb+3*\tdelay}{
% hyperbolae
\draw [bg-b, dashed, smooth, domain=1.1:\halfwidth, samples=40, variable=\tt] plot ({\tt}, {-\cte/(\tt-1)});
\draw [bg-b, dashed, smooth, domain=-\halfwidth:-1.1, samples=40, variable=\tt] plot ({\tt}, {-\cte/(\tt+1)});
}

\kfclip{\tb+3*\tdelay}{\tb+4*\tdelay}{
\draw [bg-b, dashed, smooth, domain=-pi:pi, samples=80, variable=\tt] 
plot ({0.7*cos(6*\tt r)+\ctd*cos((\tt-0.1) r)}, {0.7*sin(6*\tt r)+\ctd*sin((\tt-0.1) r)});
}

\kfclip{\tb+4*\tdelay}{\tend+1}{
\pgfkeys{/pgf/fpu=false}
\draw[dashed, bg-b] ({\cte*cos(-20+\ctoa)}, {\cte*sin(-20+\ctoa)}) arc (-20+\ctoa:\cth+\ctoa:\cte);
\pgfkeys{/pgf/fpu=true}
}

\crab{1}{\ctd*cos(\cth)}{\ctd*sin(\cth)}{90+\cth}

\draw[dashed] ({(\ctd-\cto)*cos(\cth)}, {(\ctd-\cto)*sin(\cth)}) -- ({\cto*cos(\cth)/3}, {\cto*sin(\cth)/3});

\kfclip{\tb}{\tb+4*\tdelay}{
\kftext{\cte}{9/4}{\colorbox{fg-a}{\Large {\textbf{?}}}}
}
\end{scope}
}
 % 00:23
% 00:30
%% \kfframes{1}{
   \fill[fg-a] (0, 0) rectangle (2*\halfwidth, 2*\halfheight);
}
\kfframes{2}{
   \fill[fg-a] (0, 0) rectangle (2*\halfwidth, 2*\halfheight);
   \color{white!10!black}
   \crab{1}{\halfwidth}{\halfheight}{0}
   \kftext[polar=false]{\halfwidth+1.5}{\halfheight}{?}
}
\kfframes{1}{
   \fill[fg-a] (0, 0) rectangle (2*\halfwidth, 2*\halfheight);
}




% \kfframes{4}{\ztitle{This Is A(nother) Video About Complex Numbers.}}


%% 0.2 ROOTS OF UNITY
% \kfframes{4}{\ztitle{Roots of Unity}}
% \kfframes{18}{
\newcommand\ta{6} % ONE CUBED ...
\newcommand\tb{13} % OVER THE COMPLEX ...
\newcommand\tc{17} % fadeout
\newcommand\tz{19} % end

\begin{scope}[shift={(\halfwidth, \halfheight)}]

% draw the real line and the origin

% labelled origin - in
\kftween[in=0,duration=\dt]{0}{0.12}
\pgfmathsetmacro\r\pgfmathresult
\kftween[in=\tb,duration=\dt]{0.12}{0}
\pgfmathsetmacro\r{\t < \tb ? \r : \pgfmathresult}

\fill (0, 0) circle (\r);
\kftextin[in=0, duration=\dt, out=\tb]{0.6}{-pi/4}{$0$};
\kftextout[in=\tb, out=\tb+\dt]{0.6}{-pi/4}{$0$};

\kftween[in=\tc, duration=\dt]{100}{0}
\pgfmathsetmacro\o\pgfmathresult

% axes to back
\kftween[in=0,duration=\dt]{-\halfwidth}{\halfwidth}
\pgfmathsetmacro\l\pgfmathresult
\draw[fg-a!\o!bg-b] (-\halfwidth, 0) -- (\l, 0);

\kftween[in=\tb,duration=\dt]{0}{\halfheight}
\pgfmathsetmacro\l\pgfmathresult
\draw[fg-a!\o!bg-b] (0, -\l) -- (0, \l);

% 1 - this should look as if it was drawn by a \zin, but we're not in
% a \zgrid yet, so fake it
\kftween[in=\ta,duration=(\dt/2)]{0}{2}
\pgfmathsetmacro\l\pgfmathresult
\kftween[in=(\ta+\dt/2),duration=(\dt/2)]{0}{0.12}
\pgfmathsetmacro\r\pgfmathresult
\draw (0, 0) -- (\l, 0);
\fill (2, 0) circle (\r);
\kftextin[in=\ta+\dt, duration=\dt, out=\tz, offset fst=0.6, offset snd=-pi/4]{2}{0}{$1$};

\kftextin   [polar=false, in=\dt, duration=\dt, out=\tb+\dt]{\halfwidth-0.5}{0.5}{$\mathbb{R}$}
\kftexttween[polar=false, in=(\tb+\dt), duration=\dt, out=\tz]
 {\halfwidth-0.5}{+0.5}{$\mathbb{R}$}
 {\halfwidth-0.6}{+0.5}{\textcolor{fg-a!\o!bg-b}{$\operatorname{Re}$}}
\kftextin   [polar=false, in=\tb+\dt, duration=\dt, out=\tz]{+0.6}{\halfheight-0.4}{\textcolor{fg-a!\o!bg-b}{$\operatorname{Im}$}}


%% % x^3 - 1 = 0 -> z^3 - 1 = 0
\kftextin [polar=false, in=2.0, duration=\dt, out=\tc]{0}{+2}{\colorbox{black}{$\phantom{x^3 - 1 = 0}$}}
\kftextout[polar=false, in=\tc, out=\tc+2*\dt]{0}{+2}{\colorbox{black}{$\phantom{x^3 - 1 = 0}$}}

\kftextin [polar=false, in=2.0, duration=\dt, out=\tc]{0}{+2}{$\phantom{x}^3 - 1 = 0$}
\kftextout[polar=false, in=\tc, out=\tc+\dt]{0}{+2}{$\phantom{x}^3 - 1 = 0$}


\kftextin[polar=false, in=2.0, duration=\dt, out=\tb]{0}{+2}{$x\phantom{^3 - 1 = 0}$}
%  \kftextout[polar=false, in=\tb, duration=\dt]{\halfwidth}{\halfheight+2}{\Large $x\phantom{^3 - 1 = 0}$}
%  \kftextin[polar=false, in=\tb, duration=\dt, out=\tc]{\halfwidth}{\halfheight+2}{\Large $z\phantom{^3 - 1 = 0}$}
\kftexttween[polar=false, in=\tb, duration=\dt, out=\tc]
{0}{+2}{$x\phantom{^3 - 1 = 0}$}
{0}{+2}{$z\phantom{^3 - 1 = 0}$}
\kftextout[polar=false, in=\tc, duration=\dt, out=\tc+\dt]
{0}{+2}{$z\phantom{^3 - 1 = 0}$}

\end{scope}
}

% \kfframes{10}{
  \newcommand\md{1.5}
  \newcommand\ar{0.98}
  \newcommand\delay{2.5}
  \newcommand\ta{9}
  \zgrid[unit=1.2]{

    \z{1}{0}
    \kftext[offset fst=0.6, offset snd=-pi/4]{\zunit}{0}{$1$};

    \ztween[out=\ta]{1}{0}{\md}{\ar}
    \ztween[in=\delay, out=\ta, arg origin=\ar]{\md}{\ar}{\md^2}{2*\ar}
    \ztween[in=2*\delay, out=\ta, arg origin=2*\ar]{\md^2}{2*\ar}{\md^3}{3*\ar}

    \kftextin[in=2*\dt, duration=\dt, out=\ta, offset fst=0.5, offset snd=0]{\zunit*\md}{\ar}{$z$};
    \kftextin[in=\delay+2*\dt, duration=\dt, out=\ta, offset fst=0.7, offset snd=pi/4]{\zunit*\md^2}{2*\ar}{$z^2$};
    \kftextin[in=2*\delay+2*\dt, duration=\dt, out=\ta, offset fst=0.6, offset snd=pi/2]{\zunit*\md^3}{3*\ar}{$z^3$};

    % outros
    \zout[in=\ta, out=\ta+\dt]{\md}{\ar}
    \zout[in=\ta, out=\ta+\dt, arg origin=\ar]{\md^2}{2*\ar}
    \zout[in=\ta, out=\ta+\dt, arg origin=2*\ar]{\md^3}{3*\ar}

    \kftextout[in=\ta, out=\ta+\dt, duration=\dt/2, offset fst=0.5, offset snd=0]{\zunit*\md}{\ar}{$z$};
    \kftextout[in=\ta, out=\ta+\dt, duration=\dt/2, offset fst=0.7, offset snd=pi/4]{\zunit*\md^2}{2*\ar}{$z^2$};
    \kftextout[in=\ta, out=\ta+\dt, duration=\dt/2, offset fst=0.6, offset snd=pi/2]{\zunit*\md^3}{3*\ar}{$z^3$};    
  }
}

% \kfframes{14}{ % : 46 - :00

  \newcommand\ta{0} %% BOTH WITH MODULUS ...
  \newcommand\tb{3} %% THEIR ARGUMETNS ARE ...
  \newcommand\tdelay{1.5}
  \newcommand\td{13}
  \zgrid[unit=3]{
    \zout[in=0, out=\td+\dt]{1}{0}
    \kftextout[out=\td+\dt, duration=\dt, offset fst=0.6, offset snd=-pi/4]{\zunit}{0}{$1$};

    \kftween[in=\ta, duration=\dt]{0}{2*pi}
    \pgfmathsetmacro\th\pgfmathresult
    \kftween[in=\td, duration=\dt]{2*pi}{0}
    \pgfmathsetmacro\thb\pgfmathresult
    \pgfmathsetmacro\th{\t > \td ? \pgfmathresult : \th}

    
    \pgfkeys{/pgf/fpu=false}
    \draw[dashed] (\zunit, 0) arc (0:\th r:\zunit);
    \pgfkeys{/pgf/fpu=true}

    \zin[in=\ta+\dt, out=\td, arg visible=false]{1}{2*pi/3}
    \zin[in=\ta+2*\dt, out=\td, arg visible=false]{1}{-2*pi/3}

    \zargin[in=\tb, out=(\tb+\tdelay)]{2*pi/3}
    \zargout[in=(\tb+\tdelay), out=(\tb+\tdelay+\dt)]{2*pi/3}
    \kftextin[duration=\dt, in=(\tb+\dt), out=(\tb+\tdelay)]{1.8}{pi/4}{$\frac{2}{3}\pi$}
    \kftextout[duration=\dt, in=(\tb+\tdelay), out=(\tb+\tdelay+\dt)]{1.8}{pi/4}{$\frac{2}{3}\pi$}

    \kftextin[duration=\dt, in=\tb+\tdelay, out=\td, offset fst=1, offset snd=0.9*pi]{\zunit}{2*pi/3}{$e^{i\frac{2}{3}\pi}$}

    
    \pgfmathsetmacro\tb{\tb+3}
    \zargin[in=\tb, out=(\tb+\tdelay)]{-2*pi/3}
    \zargout[in=(\tb+\tdelay), out=(\tb+\tdelay+\dt)]{-2*pi/3}
    \kftextin[duration=\dt, in=(\tb+\dt), out=(\tb+\tdelay)]{1.8}{-pi/4}{$-\frac{2}{3}\pi$}
    \kftextout[duration=\dt, in=(\tb+\tdelay), out=(\tb+\tdelay+\dt)]{1.8}{-pi/4}{$-\frac{2}{3}\pi$}

    \kftextin[duration=\dt, in=\tb+\tdelay, out=\td, offset fst=1.3, offset snd=1.1*pi]{\zunit}{-2*pi/3}{$e^{-i\frac{2}{3}\pi}$}

    \kftextout[duration=\dt, in=\td, out=\td+\dt, offset fst=1, offset snd=0.9*pi]{\zunit}{2*pi/3}{$e^{i\frac{2}{3}\pi}$}
    \kftextout[duration=\dt, in=\td, out=\td+\dt, offset fst=1.3, offset snd=1.1*pi]{\zunit}{-2*pi/3}{$e^{-i\frac{2}{3}\pi}$}

    \zout[in=\td, out=\td+\dt, arg visible=false]{1}{2*pi/3}
    \zout[in=\td, out=\td+\dt, arg visible=false]{1}{-2*pi/3}

    
  }
}

% \kfframes{19}{ % : 00 - : 19
\newcommand\ta{4} % REMEMBER THAT ...

\newcommand\tb{7} % first z out
\newcommand\tc{8}

\newcommand\td{11} % second arg in
\newcommand\targdelay{1}
\newcommand\te{14} % r in
\newcommand\tf{16}
\newcommand\tg{18.5}


\newcommand\zr{1}
\newcommand\za{1}
%% \newcommand\tc{6}
%% \newcommand\td{9}

\zgrid[unit=3]{
\pgfmathsetmacro\zx{\zunit*\zr*cos(\za r)}
\pgfmathsetmacro\zy{\zunit*\zr*sin(\za r)}


\zargin[in=\ta+\dt, out=(\tb)]{\za}
\zin[in=\ta, out=\tb, arg visible=false]{\zr}{\za}

\kfclip{\tb}{\tend}{
\fill[bg-b] (\zx, \zy) circle (0.12);
}

\zout[in=\tb, out=\tc, arg visible=false]{\zr}{\za}
\zargout[in=\tb, out=\tc)]{\za}

\kftextin [in=\ta+\dt, duration=\dt, out=\tb, offset fst=0.3, offset snd=(\zr+pi/2)]{0.7*\zunit}{\za}{$r$}
\kftextin [in=\ta+2*\dt, duration=\dt, out=\tb]{1.5}{0.4}{$\theta$}

\kftextout[in=\tb, duration=\dt, out=\tb+\dt, offset fst=0.3, offset snd=(\zr+pi/2)]{0.7*\zunit}{\za}{$r$}
\kftextout[in=\tb, duration=\dt, out=\tb+\dt]{1.5}{0.4}{$\theta$}

\kftextin[in=\tb, duration=\dt, out=\tf, offset fst=1.7, offset snd=0.2]{\zr*\zunit}{\za}{$e^{i\theta\phantom{(2\pi+\theta)}}$}


\kftexttween[in=\tf, duration=\dt, out=\tg, offset fst=1.7, offset snd=0.2]
{\zr*\zunit}{\za}{$e^{i\theta\phantom{(2\pi+\theta)}}$}
{\zr*\zunit}{\za}{$e^{i(2\pi+\theta)\phantom{\theta}}$}

\kftextout[in=\tg, duration=\dt, out=\tg+\dt, offset fst=1.7, offset snd=0.2]{\zr*\zunit}{\za}{$e^{i(2\pi+\theta)\phantom{\theta}}$}

% second z
\zargin[in=\td, out=\tf]{2*pi}
\zargin[in=\td+\targdelay, out=\tf, arg radius=1.3]{\za}

%\kftextin [in=\td, duration=\dt, out=\te, offset fst=0.3, offset snd=(\zr+pi/2)]{0.5*\zunit}{\za}{$r$}
\kftextin [in=\td+\dt, duration=\dt, out=\tf]{1.6}{2.9}{$2\pi$}
\kftextin [in=\td+\targdelay+\dt, duration=\dt, out=\tf]{1.8}{0.4}{$\theta$}

\zargout[in=\tf, out=\tf+\dt)]{2*pi}
\zargout[in=\tf, out=\tf+\dt), arg radius=1.3]{\za}


\kftextout[in=\tf, duration=\dt, out=\tf+\dt]{1.6}{2.9}{$2\pi$}
\kftextout[in=\tf, duration=\dt, out=\tf+\dt]{1.8}{0.4}{$\theta$}


\zin[in=\te, out=\tg, arg visible=false]{\zr}{\za}
\zout[in=\tg, out=\tg+\dt, arg visible=false]{\zr}{\za}

\kftextin [in=\te, duration=\dt, out=\tf, offset fst=0.3, offset snd=(\zr+pi/2)]{0.7*\zunit}{\za}{$r$}
\kftextout[in=\tf, duration=\dt, out=\tf+\dt, offset fst=0.3, offset snd=(\zr+pi/2)]{0.7*\zunit}{\za}{$r$}

}
}

% \kfframes{8}{                                     %% : 19 - : 27
  \newcommand\ta{2}
  \newcommand\tb{4}
  \newcommand\tc{6}

  \zgrid[unit=3]{
    \zin[arg visible=false, out=\tend+1]{1}{0}
    \zin[arg visible=false, in=\dt, out=\tend+1]{1}{2*pi/3}
    \zin[arg visible=false, in=2*\dt, out=\tend+1]{1}{-2*pi/3}


    \kftween[in=\dt, duration=\dt]{0}{2*pi}
    \pgfmathsetmacro\th\pgfmathresult
    \pgfkeys{/pgf/fpu=false}
    \draw[dashed] (\zunit, 0) arc (0:\th r:\zunit);
    \pgfkeys{/pgf/fpu=true}

    % 3 roots in
    \kftextin[in=\ta, duration=\dt, out=\tc]{1}{pi/2}{\colorbox{bg-a}{\phantom{Third roots of unity.}}}
    \kftextin[in=\tb, duration=\dt, out=\tc, polar=false]{-2.6}{0.45}{\colorbox{bg-a}{\phantom{\scriptsize (resp. $n$-th)}}}

    \kftextin[in=(\ta+\dt), duration=\dt, out=(\tc-\dt)]{1}{pi/2}{Third roots of unity.}
    \kftextin[in=(\tb+\dt), duration=\dt, out=(\tc-\dt), polar=false]{-2.6}{0.45}{\footnotesize (resp. $n$-th)}

    % out
    \kftextout[in=\tc, duration=\dt, out=\tc+\dt]{1}{pi/2}{\colorbox{bg-a}{\phantom{Third roots of unity.}}}
    \kftextout[in=\tc, duration=\dt, out=\tc+\dt, polar=false]{-2.6}{0.45}{\colorbox{bg-a}{\phantom{\scriptsize (resp. $n$-th)}}}

    \kftextout[in=(\tc-\dt), duration=\dt, out=\tc]{1}{pi/2}{Third roots of unity.}
    \kftextout[in=(\tc-\dt), duration=\dt, out=\tc, polar=false]{-2.6}{0.45}{\footnotesize (resp. $n$-th)}   
  }
}

% \kfframes{5}{
  \newcommand\ta{1} % replace circle with triangle
  \newcommand\tdelay{0.5}
  \newcommand\tb{4} % triangle goes.
  \zgrid[unit=3]{
    \zout[out=\tb+\dt, arg visible=false]{1}{0}
    \zout[out=\tb+\dt, arg visible=false]{1}{2*pi/3}
    \zout[out=\tb+\dt, arg visible=false]{1}{-2*pi/3}


    \kftween[in=\ta, duration=\dt]{0}{2*pi}
    \pgfmathsetmacro\th\pgfmathresult
    
    \pgfkeys{/pgf/fpu=false}
    \draw[dashed] ({\zunit*cos(\th r)}, {\zunit*sin(\th r)}) arc (\th r:2*pi r:\zunit);
    \pgfkeys{/pgf/fpu=true}

    \kftextin[in=\ta+2*\dt, duration=\dt, out=\tb, polar=false]{\halfwidth-4}{-\halfheight+2}{
    \scriptsize \begin{tabular}{r} *perhaps better known\\as an \emph{equilateral} triangle.\end{tabular}
    }
    \kftextout[in=\tb, duration=\dt, out=\tb+\dt, polar=false]{\halfwidth-4
    }{-\halfheight+2}{
    \scriptsize \begin{tabular}{r} *perhaps better known\\as an \emph{equilateral} triangle.\end{tabular}
    }

    \pgfmathsetmacro\ta{\ta+\tdelay}

    \zngon[in=\ta, out=(\tb+\dt)]{3}
  }
}

% \kfframes{19}{
  \newcommand\ta{0} % fourth roots in
  \newcommand\tb{3} % quadrilateral in
  \newcommand\tbb{4}
  \newcommand\tstay{3}
  \newcommand\tc{8} % labels in
  \newcommand\tdelay{1}
  \newcommand\td{17} % everything out
  
  \zgrid[unit=3]{
    \zin[out=\td, arg visible=false]{1}{0}
    \zin[in=(\dt/4),   out=\td, arg visible=false]{1}{pi/2}
    \zin[in=(\dt/2),   out=\td, arg visible=false]{1}{pi}
    \zin[in=(3*\dt/4), out=\td, arg visible=false]{1}{3*pi/2}

    \kftween[in=(\ta+\dt), duration=\dt]{0}{2*pi}
    \pgfmathsetmacro\tha\pgfmathresult
    
    \kftween[in=(\tb-\dt), duration=\dt]{0}{2*pi}
    \pgfmathsetmacro\thb\pgfmathresult
    
    \pgfkeys{/pgf/fpu=false}
    \draw[dashed] ({\zunit*cos(\thb r)}, {\zunit*sin(\thb r)}) arc (\thb r:\tha r:\zunit);
    \pgfkeys{/pgf/fpu=true}

    \kftextin[in=\tbb, duration=\dt, out=\tbb+\tstay, polar=false]{\halfwidth-3}{-\halfheight+2}{
    \scriptsize \begin{tabular}{r} *better known\\as a \emph{square.}\end{tabular}
    }
    \kftextout[in=\tbb+\tstay, duration=\dt, out=\tbb+\tstay+\dt, polar=false]{\halfwidth-3}{-\halfheight+2}{
    \scriptsize \begin{tabular}{r} *better known\\as a \emph{square.}\end{tabular}
    }

    % labels
    \kftextin[in=\tc,             out=\td, duration=\dt, offset fst=0.6, offset snd=-pi/4]{\zunit}{0}{$1$}
    \kftextin[in=(\tc+\tdelay),   out=\td, duration=\dt, offset fst=0.6, offset snd=3*pi/4]{\zunit}{pi/2}{$i$}
    \kftextin[in=(\tc+2*\tdelay), out=\td, duration=\dt, offset fst=0.8, offset snd=2.6]{\zunit}{pi}{$-1$}
    \kftextin[in=(\tc+3*\tdelay), out=\td, duration=\dt, offset fst=0.7, offset snd=-0.5]{\zunit}{-pi/2}{$-i$}

    \zout[in=\td, out=(\td+\dt), arg visible=false]{1}{pi/2}
    \zout[in=\td, out=(\td+\dt), arg visible=false]{1}{pi}
    \zout[in=\td, out=(\td+\dt), arg visible=false]{1}{3*pi/2}
    
    \kftextout[in=\td, out=(\td+\dt), offset fst=0.6, offset snd=-pi/4]{\zunit}{0}{$1$}
    \kftextout[in=\td, out=(\td+\dt), offset fst=0.6, offset snd=3*pi/4]{\zunit}{pi/2}{$i$}
    \kftextout[in=\td, out=(\td+\dt), offset fst=0.8, offset snd=2.6]{\zunit}{pi}{$-1$}
    \kftextout[in=\td, out=(\td+\dt), offset fst=0.7, offset snd=-0.5]{\zunit}{-pi/2}{$-i$}

    
    \zngon[in=\tb, out=(\td-\dt)]{4}
  }
}

% \kfframes{7}{
  \newcommand\tdelay{2.5}

  \pgfmathtruncatemacro\dn{\t / \tdelay}
  \pgfmathtruncatemacro\n{5+\dn}
  \pgfmathtruncatemacro\m{\n-1}
  \pgfmathsetmacro\ta{\tdelay*\dn}
  \pgfmathsetmacro\tb{\tdelay*(\dn+1)}

  \zgrid[unit=3]{
    \foreach \j in {0, ..., \m}{
      \zin[in=(\ta+\j*\dt/\n), out=(\tb-\dt), arg visible=false]{1}{\j*2*pi/\n}
      \zout[in=(\tb-\dt), out=\tb, arg visible=false]{1}{\j*2*pi/\n}
    }
    \zngon[in=\ta+\dt, out=\tb]{\n}
  }
}

% \kfframes{8}{
  \newcommand\ta{0} % fade out the 7-gon
  \newcommand\tb{4}
  \newcommand\tc{5}
  \newcommand\td{7}

  
  \zgrid[unit=3]{
    \kftween[in=\ta, duration=\dt]{100}{0}
    \colorlet{xc}{fg-a!\pgfmathresult!bg-b}
    
    \begin{scope}[color=xc]
      \foreach \j in {0,..., 6}{ % feckit, hardcode this.
        \fill ({\zunit*cos(\j*2*pi/7 r)}, {\zunit*sin(\j*2*pi/7 r)}) circle (0.12);
        \zout[out=\ta, arg visible=false]{1}{\j*2*pi/7}
      }

      \zngon[in=-\dt, out=\tend+\dt]{7}
    \end{scope}

    \zin[in=\tb, out=\tc, arg visible=false]{1}{2*pi/7}
    \zout[in=\tc, out=\tc+\dt, arg visible=false]{1}{2*pi/7}

    \kftextin[in=\tb, out=\tc, duration=\dt, offset fst=0.6, offset snd=pi/4]{\zunit}{2*pi/7}{$z$}
    \kftextout[in=\tc, out=\tc+\dt, offset fst=0.6, offset snd=pi/4]{\zunit}{2*pi/7}{$z$}

    \pgfmathsetmacro\xzx{\zunit*cos(2*pi/7 r)}
    \pgfmathsetmacro\xzy{\zunit*sin(2*pi/7 r)}

    \kftween[in=\tc, duration=\dt]{\xzy}{-\xzy}
    \pgfmathsetmacro\xzya\pgfmathresult

    \kftween[in=\tc+\dt, duration=\dt]{\xzy}{-\xzy}
    \pgfmathsetmacro\xzyb\pgfmathresult
    
    \kftween[in=\tc+\dt, duration=\dt]{0}{0.12}
    \pgfmathsetmacro\xzr\pgfmathresult

    \draw[dashed] (\xzx, \xzya) -- (\xzx, \xzyb);
    \fill (\xzx, -\xzy) circle (\xzr);

    \zin[in=\td, arg visible=false]{1}{-2*pi/7}
    \kftextin[in=\tc+\dt, duration=\dt, offset fst=0.6, offset snd=-pi/4]{\zunit}{-2*pi/7}{$\bar{z}$}

  }

}


%% 0.3 WE ARE GOING TO PROVE A THEOREM
% \kfframes{4}{\ztitle{We Are Going to Prove a Theorem}

% polarshift tool
\newenvironment{polarshift}[2]{
  \begin{scope}[shift={({\zunit*(#1)*cos(#2 r)}, {\zunit*(#1)*sin(#2 r)})}]
}{
  \end{scope}
}
% \kfframes{9}{
  \newcommand{\ta}{1}
  \newcommand{\tb}{2} % grid in
  \newcommand{\tc}{3} % z in
  \newcommand{\td}{6} % where is P(z) ?
  \newcommand{\tda}{8} % z out
  \newcommand{\te}{8}
  
  \pgfmathsetmacro\za{1.3}
  \pgfmathsetmacro\zr{1.3}
  
  \zgrid[in=\tb, out=\tend+1, origin y=\halfheight-1.5, unit=1]{

    \zin[in=\tc, out=\tda, arg visible=false]{\zr}{\za}
    \zout[in=\tda, duration=\dt, out=\tda+\dt, arg visible=false]{\zr}{\za}

\kftextin[in=\tc, out=\tda, duration=\dt, offset fst=0.55, offset snd=pi/8]{\zunit*\zr}{\za}{$z$}
\kftextout[in=\tda, out=\tda+\dt, duration=\dt, offset fst=0.55, offset snd=pi/8]{\zunit*\zr}{\za}{$z$}

    % \z[arg visible=false]{\zr^3}{3*\za}
    \begin{polarshift}{\zr^3}{3*\za}
      % \z[arg visible=false]{-3*\zr^2}{2*\za}
      \begin{polarshift}{-3*\zr^2}{2*\za}
        % \z[arg visible=false]{4*\zr}{\za}
        \begin{polarshift}{4*\zr}{\za}
          %\z[arg visible=false]{2}{pi}
          \begin{polarshift}{2}{pi}
            \kftween[in=(\td-\dt/2), duration=(\dt/2)]{0}{0.12}
            \pgfmathsetmacro\xr\pgfmathresult
            \kftween[in=(\te-\dt/2), duration=(\dt/2)]{0}{0.12}
            \pgfmathsetmacro\xr{\xr-\pgfmathresult}

            \fill (0, 0) circle (\xr);
            \kftextin[in=\td, duration=\dt, out=(\te-\dt)]{1.5}{0}{$P(z)$ ?}
            \kftextout[in=(\te-\dt), out=\te]{1.5}{0}{$P(z)$ ?}
         
          \end{polarshift}          
        \end{polarshift}
      \end{polarshift}
    \end{polarshift}   
  }
  \kftween[in=\ta, duration=\dt]{11}{\halfwidth}
  \pgfmathsetmacro\xw\pgfmathresult
 
  \kftextin[duration=\dt, out=\tend+1, polar=false]{\xw}{\halfheight+2}{\colorbox{bg-a}{$P(z) \;\phantom{= z^3 - 3z^2 + 4z - 2}$}}
  \kftextin[in=(\ta+\dt/2), out=\tend+1, duration=\dt, polar=false]{\halfwidth}{\halfheight+2}{$\phantom{P(z)} = z^3 - 3z^2 + 4z - 2$}

}

% \kfframes{11}{
  
  \pgfmathsetmacro\za{1.3}
  \pgfmathsetmacro\zr{1.3}

  \newcommand{\ta}{1}
  \newcommand{\tdelay}{0.5}
  \newcommand{\targdelay}{0.5}
  \newcommand{\tshdelay}{(\dt/2)}
  \newcommand{\taa}{0} % pan
  \newcommand{\tb}{4} % scaling
  \newcommand{\tc}{7} % shifting
  \newcommand{\td}{8} % polynomial text
  \newcommand{\te}{10} % P (z)

  % pan camera
  
  \zgrid[out=\tend+1, origin y=\halfheight-1.5, unit=1]{
    % 1
    \kftextin[in=\ta, out=\tb, duration=\dt, offset fst=0.6, offset snd=-pi/4]{1}{0}{$1$}
    \kftextout[in=\tb, out=(\tb+\dt), duration=\dt, offset fst=0.6, offset snd=-pi/4]{1}{0}{$1$} 
    \kftextin[in=\tb+\dt, out=\td, duration=\dt, offset fst=0.8, offset snd=5*pi/6]{-2*\zunit}{0}{$-2$}
    \kftextout[in=\td, out=\td+\dt, duration=\dt, offset fst=0.8, offset snd=5*pi/6]{-2*\zunit}{0}{$-2$}

    \zin[in=\ta, out=\tb, arg visible=false]{1}{0}
    \ztween[in=\tb, out=\tend+1, arg visible=false, stagger=false]{1}{0}{-2}{0}

    \kftween[in=\tc, duration=\dt]{0}{1} \pgfmathsetmacro\xpsa\pgfmathresult
    \begin{polarshift}{-2*\xpsa}{0}
      % z
      
      \kftextin[in=\ta+\tdelay, out=\tb+\tdelay, duration=\dt, offset fst=0.5, offset snd=pi/6]{\zunit*\zr}{\za}{$z$}
      \kftextout[in=\tb+\tdelay, out=(\tb+\tdelay+\dt), duration=\dt, offset fst=0.5, offset snd=pi/6]{\zunit*\zr}{\za}{$z$} 
      \kftextin[in=\tb+\tdelay+\dt, out=\td, duration=\dt, offset fst=0.7, offset snd=pi/6]{\zunit*4*\zr}{\za}{\colorbox{bg-a}{$4z$}}
      \kftextout[in=\td, duration=\dt, out=(\td+\dt), offset fst=0.7, offset snd=pi/6]{\zunit*4*\zr}{\za}{\colorbox{bg-a}{$4z$}}

      \ztween[in=\ta+\tdelay, out=\tb+\tdelay, arg visible=false, stagger=false]{1}{0}{\zr}{\za}
      \ztween[in=\tb+\tdelay, out=\tend+1, arg visible=false, stagger=false]{\zr}{\za}{4*\zr}{\za}

      \zargin[in=\ta+\tdelay, out=\ta+\targdelay]{\za}
      \zargout[in=\ta+\tdelay+\targdelay, out=\ta+\tdelay+\targdelay+\dt]{\za}

    \end{polarshift}

    \kftween[in=\tc+\tshdelay, duration=\dt]{0}{1} \pgfmathsetmacro\xpsb\pgfmathresult
    \begin{polarshift}{-2*\xpsb}{0}
      \begin{polarshift}{\xpsb*4*\zr}{\xpsb*\za}
        % z^2
        
        \kftextin[in=\ta+2*\tdelay, out=\tb+2*\tdelay, duration=\dt, offset fst=0.6, offset snd=pi/2]{\zunit*\zr^2}{2*\za}{$z^2$}
        \kftextout[in=\tb+2*\tdelay, out=(\tb+2*\tdelay+\dt), duration=\dt, offset fst=0.6, offset snd=pi/2]{\zunit*\zr^2}{2*\za}{$z^2$}
        \kftextin[in=\tb+2*\tdelay+\dt, out=\td, duration=\dt, offset fst=0.8, offset snd=(pi/4-pi*\xpsb/2)]{-3*\zunit*\zr^2}{2*\za}{$-3z^2$}
        \kftextout[in=\td, out=\td+\dt, duration=\dt, offset fst=0.8, offset snd=(pi/4-pi*\xpsb/2)]{-3*\zunit*\zr^2}{2*\za}{$-3z^2$}

        \ztween[in=\ta+2*\tdelay, out=\tb+2*\tdelay, arg visible=false, stagger=false]{\zr}{\za}{\zr^2}{2*\za}
        \ztween[in=\tb+2*\tdelay, out=\tend+1, arg visible=false, stagger=false]{\zr^2}{2*\za}{-3*\zr^2}{2*\za}

        \zargin[in=\ta+2*\tdelay, arg origin=\za,  out=\ta+2*\tdelay+\targdelay]{2*\za}
        \zargout[in=\ta+2*\tdelay+\targdelay, arg origin=\za, out=\ta+2*\tdelay+\targdelay+\dt]{2*\za}

      \end{polarshift}
    \end{polarshift}

    % z^3
    \kftween[in=\tc+2*\tshdelay, duration=\dt]{0}{1} \pgfmathsetmacro\xpsc\pgfmathresult
    \begin{polarshift}{-2*\xpsc}{0}
      \begin{polarshift}{\xpsc*4*\zr}{\xpsc*\za}
        \begin{polarshift}{\xpsc*-3*\zr^2}{\xpsc*2*\za}

          \kftextin[in=\ta+3*\tdelay+\dt/2, out=\td, duration=\dt, offset fst=0.6, offset snd=5*pi/6]{\zunit*\zr^3}{3*\za}{$z^3$}
          \kftextout[in=\td, duration=\dt, out=\td+\dt, offset fst=0.6, offset snd=5*pi/6]{\zunit*\zr^3}{3*\za}{$z^3$}

          \kftextin[in=\td, duration=\dt, out=\te+\dt, offset fst=1.1, offset snd=-pi/6]{\zunit*\zr^3}{3*\za}{\colorbox{bg-a}{\phantom{$z^3-3z^2+4z-2$}}}
          \kftextin[in=\td+\dt, duration=\dt, out=\te, offset fst=1.1, offset snd=-pi/6]{\zunit*\zr^3}{3*\za}{$z^3-3z^2+4z-2$}

          \kftexttween[in=\te+\dt, duration=\dt, out=\tend+1, offset fst=1.1, offset snd=-pi/6]{\zunit*\zr^3}{3*\za}{\colorbox{bg-a}{\phantom{$z^3-3z^2+4z-2$}}}{\zunit*\zr^3}{3*\za}{\colorbox{bg-a}{\phantom{$P(z)$}}}

          
          \kftexttween[in=\te, duration=\dt, out=\tend+1, offset fst=1.1, offset snd=-pi/6]{\zunit*\zr^3}{3*\za}{$z^3-3z^2+4z-2$}{\zunit*\zr^3}{3*\za}{$P(z)$}

          
          \ztween[in=\ta+3*\tdelay, out=\tb+3*\tdelay, arg visible=false, stagger=false]{\zr^2}{2*\za}{\zr^3}{3*\za}
          \ztween[in=\tb+3*\tdelay, out=\tend+1, arg visible=false, stagger=false]{\zr^3}{3*\za}{\zr^3}{3*\za}

          \zargin[in=\ta+3*\tdelay, arg origin=2*\za, out=\ta+3*\tdelay+\targdelay]{3*\za}
          \zargout[in=\ta+3*\tdelay+\targdelay, arg origin=2*\za, out=\ta+3*\tdelay+\targdelay+\dt]{3*\za}


        \end{polarshift}
      \end{polarshift}
    \end{polarshift}

  }

  \kftextout[out=2*\dt, duration=\dt, polar=false]{\halfwidth}{\halfheight+2}{\colorbox{bg-a}{\phantom{$P(z) = z^3 - 3z^2 + 4z - 2$}}}
  \kftextout[out=\dt, duration=\dt, polar=false]{\halfwidth}{\halfheight+2}{$P(z) = z^3 - 3z^2 + 4z - 2$}

}

% \kfframes{4}{

  \newcommand\ta{1} % z in
  \newcommand\tb{2} % z bar
  \newcommand\tc{2.5} % pan
  
  \pgfmathsetmacro\za{1.3}
  \pgfmathsetmacro\zr{1.3}

  % dim
  \kftween[in=0, duration=\dt]{0}{100}
  \pgfmathsetmacro\zxc\pgfmathresult

  
  % pan camera
  \kftween[in=\tc, duration=\dt]{-1.5}{1}
  \pgfmathsetmacro\zyo\pgfmathresult
  
  \zgrid[out=\tend+1, origin y=\halfheight+\zyo, unit=1]{

    \begin{scope}[color=bg-b!\zxc!fg-a]
    \z[out=\tend+1, arg visible=false]{-2}{0}
    \begin{polarshift}{-2}{0}
      % z
      \z[out=\tend+1, arg visible=false]{4*\zr}{\za}
      \begin{polarshift}{4*\zr}{\za}
        % z^2
        \z[out=\tend+1, arg visible=false]{-3*\zr^2}{2*\za}
        \begin{polarshift}{-3*\zr^2}{2*\za}          
          % z^3
          \kftext[offset fst=1.1, offset snd=-pi/6]{\zunit*\zr^3}{3*\za}{\colorbox{bg-a}{$P(z)$}}

          \z[out=\tend+1, arg visible=false]{\zr^3}{3*\za}
          
        \end{polarshift}
      \end{polarshift}
    \end{polarshift}
    \end{scope}

    \zin[in=\ta, out=\tb, arg visible=false]{\zr}{\za}
    \zout[in=\tb, out=\tb+\dt, arg visible=false]{\zr}{\za}

    \pgfmathsetmacro\zcx{\zunit*\zr*cos(\za r)}
    \pgfmathsetmacro\zcy{\zunit*\zr*sin(\za r)}

    \kftween[in=\tb, duration=\dt]{\zcy}{-\zcy}
    \pgfmathsetmacro\zcya\pgfmathresult
    \kftween[in=\tb+\dt, duration=\dt]{\zcy}{-\zcy}
    \pgfmathsetmacro\zcyb\pgfmathresult

    \kftween[in=\tb+\dt, duration=\dt]{0}{0.12}
    \pgfmathsetmacro\zcd\pgfmathresult

    
    \draw[dashed] (\zcx, \zcya) -- (\zcx, \zcyb);
    \fill (\zcx, -\zcy) circle (\zcd);

    \zin[in=\tb+2*\dt, out=\tend+1, arg visible=false]{\zr}{-\za}
    \kftextin[in=\tb+2*\dt, duration=\dt, offset fst=0.5]{\zr*\zunit}{-\za}{$\bar{z}$}
  }
}

% \kfframes{8}{
  
  \pgfmathsetmacro\za{1.3}
  \pgfmathsetmacro\zr{1.3}

  \newcommand{\ta}{0}
  \newcommand{\tdelay}{0.5}
  \newcommand{\tscdelay}{0}
  \newcommand{\targdelay}{0.5}
  \newcommand{\tshdelay}{0}
  \newcommand{\taa}{0} % pan
  \newcommand{\tb}{3} % scaling
  \newcommand{\tc}{4} % shifting
  \newcommand{\td}{5} % polynomial text
%  \newcommand{\te}{7} % P (z)
  
  \zgrid[out=\tend+1, origin y=\halfheight+1, unit=1]{


    % leftovers
    \begin{scope}[color=bg-b]
    \z[out=\tend+1, arg visible=false]{-2}{0}
    \begin{polarshift}{-2}{0}
      \z[out=\tend+1, arg visible=false]{4*\zr}{\za}
      \begin{polarshift}{4*\zr}{\za}
        \z[out=\tend+1, arg visible=false]{-3*\zr^2}{2*\za}
        \begin{polarshift}{-3*\zr^2}{2*\za}          
          \kftext[offset fst=1.1, offset snd=-pi/6]{\zunit*\zr^3}{3*\za}{\colorbox{bg-a}{$P(z)$}}
          \z[out=\tend+1, arg visible=false]{\zr^3}{3*\za}
        \end{polarshift}
      \end{polarshift}
    \end{polarshift}
    \end{scope}
    
  % conjugate.
  \pgfmathsetmacro\za{-1.3}

    % 1
    \kftextin[in=\ta, out=\tb, duration=\dt, offset fst=0.6, offset snd=-pi/4]{1}{0}{$1$}
    \kftextout[in=\tb, out=(\tb+\dt), duration=\dt, offset fst=0.6, offset snd=-pi/4]{1}{0}{$1$} 
    \kftextin[in=\tb+\dt, out=\td, duration=\dt, offset fst=0.8, offset snd=5*pi/6]{-2*\zunit}{0}{$-2$}
    \kftextout[in=\td, out=\td+\dt, duration=\dt, offset fst=0.8, offset snd=5*pi/6]{-2*\zunit}{0}{$-2$}

    \zin[in=\ta, out=\tb, arg visible=false]{1}{0}
    \ztween[in=\tb, out=\tend+1, arg visible=false, stagger=false]{1}{0}{-2}{0}

    \kftween[in=\tc, duration=\dt]{0}{1} \pgfmathsetmacro\xpsa\pgfmathresult
    \begin{polarshift}{-2*\xpsa}{0}
      % z
      \kftween[in=\ta, duration=\dt]{0}{-pi/3}
      \pgfmathsetmacro\zlos\pgfmathresult
      \kftextout[out=(\tb+\tscdelay+\dt), duration=\dt, offset fst=0.5, offset snd=\zlos]{\zunit*\zr}{\za}{$\bar{z}$} 

      \z[arg visible=false, out=\tb+\tscdelay]{\zr}{\za}
      \ztween[in=\ta+\tdelay, out=\ta+\tdelay+\dt, arg visible=false, stagger=false]{1}{0}{\zr}{\za}
      \ztween[in=\tb+\tscdelay, out=\tend+1, arg visible=false, stagger=false]{\zr}{\za}{4*\zr}{\za}

      \kftextin[in=\tb+\tscdelay+\dt, out=(\td+\tshdelay), duration=\dt, offset fst=0.8+0.1*\xpsa, offset snd=(pi/6+\xpsa*3*pi/4)]{\zunit*4*\zr}{\za}{\colorbox{bg-a}{$4\bar{z}$}}
      \kftextout[in=\td+\tshdelay, duration=\dt, out=(\td+\tshdelay+\dt), offset fst=0.8+0.1*\xpsa, offset snd=(pi/6+\xpsa*3*pi/4)]{\zunit*4*\zr}{\za}{\colorbox{bg-a}{$4\bar{z}$}}
      
      \zargin[in=\ta+\tdelay, out=\ta+\tdelay+\targdelay]{\za}
      \zargout[in=\ta+\tdelay+\targdelay, out=\ta+\tdelay+\targdelay+\dt]{\za}

    \end{polarshift}

    \kftween[in=\tc+\tshdelay, duration=\dt]{0}{1} \pgfmathsetmacro\xpsb\pgfmathresult
    \begin{polarshift}{-2*\xpsb}{0}
      \begin{polarshift}{\xpsb*4*\zr}{\xpsb*\za}
        % z^2
        
        \kftextin[in=\ta+2*\tdelay, out=\tb+2*\tscdelay, duration=\dt, offset fst=0.6, offset snd=-pi/2]{\zunit*\zr^2}{2*\za}{$\bar{z}^2$}
        \kftextout[in=\tb+2*\tscdelay, out=(\tb+2*\tscdelay+\dt), duration=\dt, offset fst=0.6, offset snd=-pi/2]{\zunit*\zr^2}{2*\za}{$\bar{z}^2$}
        \kftextin[in=\tb+2*\tscdelay+\dt, out=\td+2*\tshdelay, duration=\dt, offset fst=0.8, offset snd=-pi/4]{-3*\zunit*\zr^2}{2*\za}{$-3\bar{z}^2$}
        \kftextout[in=\td+2*\tshdelay, out=\td+2*\tshdelay+\dt, duration=\dt, offset fst=0.8, offset snd=-pi/4]{-3*\zunit*\zr^2}{2*\za}{$-3\bar{z}^2$}

        \ztween[in=\ta+2*\tdelay, out=\tb+2*\tscdelay, arg visible=false, stagger=false]{\zr}{\za}{\zr^2}{2*\za}
        \ztween[in=\tb+2*\tscdelay, out=\tend+1, arg visible=false, stagger=false]{\zr^2}{2*\za}{-3*\zr^2}{2*\za}

        \zargin[in=\ta+2*\tdelay, arg origin=\za,  out=\ta+2*\tdelay+\targdelay]{2*\za}
        \zargout[in=\ta+2*\tdelay+\targdelay, arg origin=\za, out=\ta+2*\tdelay+\targdelay+\dt]{2*\za}

      \end{polarshift}
    \end{polarshift}

    % z^3
    \kftween[in=\tc+2*\tshdelay, duration=\dt]{0}{1} \pgfmathsetmacro\xpsc\pgfmathresult
    \begin{polarshift}{-2*\xpsc}{0}
      \begin{polarshift}{\xpsc*4*\zr}{\xpsc*\za}
        \begin{polarshift}{\xpsc*-3*\zr^2}{\xpsc*2*\za}

          \kftextin[in=\ta+3*\tdelay+\dt/2, out=\td+3*\tshdelay, duration=\dt, offset fst=0.6, offset snd=5*pi/6]{\zunit*\zr^3}{3*\za}{$\bar{z}^3$}
          \kftextout[in=\td+3*\tshdelay, duration=\dt, out=\td+3*\tshdelay+\dt, offset fst=0.6, offset snd=5*pi/6]{\zunit*\zr^3}{3*\za}{$\bar{z}^3$}

          \kftextin[in=\td+4*\tshdelay, out=\tend+1, duration=\dt, out=\tend+1, offset fst=1.1, offset snd=-5*pi/6]{\zunit*\zr^3}{3*\za}{$P(\bar{z})$}
          \ztween[in=\ta+3*\tdelay, out=\tb+3*\tscdelay, arg visible=false, stagger=false]{\zr^2}{2*\za}{\zr^3}{3*\za}
          \ztween[in=\tb+3*\tscdelay, out=\tend+1, arg visible=false, stagger=false]{\zr^3}{3*\za}{\zr^3}{3*\za}

          \zargin[in=\ta+3*\tdelay, arg origin=2*\za, out=\ta+3*\tdelay+\targdelay]{3*\za}
          \zargout[in=\ta+3*\tdelay+\targdelay, arg origin=2*\za, out=\ta+3*\tdelay+\targdelay+\dt]{3*\za}


        \end{polarshift}
      \end{polarshift}
    \end{polarshift}

  }

}


% 0.4 BUT WHAT ABOUT THE CRAB, THOUGH?
% \kfframes{4}{\ztitle{Okay, But What About the Crab, Though?}}
% \kfframes{18}{
  \newcommand\ta{0} % line in
  \newcommand\tb{2} % 1 in
  \newcommand\tc{4} % displacement
  \newcommand\td{7} % velocity
  \newcommand\tdelay{1.5}
  \newcommand\ts{12} % disappear

\newcommand\te{14} % disappear

  
  \kftextout[polar=false, duration=\dt, out=(\te+\dt)]{\halfwidth}{6}{$\begin{aligned}\frac{df}{dt} &= \phantom{c} f\\f(0) &= 1\end{aligned}$}
  \kftext[polar=false]{\halfwidth}{6}{$\begin{aligned}\phantom{\frac{df}{dt}} &\phantom{=} \;\:c \phantom{f}\\\phantom{f(0) }&\phantom{= 1}\end{aligned}$}

  \begin{scope}[shift={({\halfwidth-3}, {\halfheight-2})}]
    \pgfmathsetmacro\zunit{3}

    %  \kftext[polar=false]{0}{3.5}{$\begin{aligned}\frac{df}{dt} = c f\\f(0) = 1\end{aligned}$}

    \kftween[in=\ta, duration=\dt]{0}{1} 
    \pgfmathsetmacro\lx\pgfmathresult
    \kftween[in=(\tb+\dt), duration=\dt]{0}{1} 
    \pgfmathsetmacro\lxp\pgfmathresult

    % axis
    \draw[bg-b] (-\halfwidth+3, 0) -- ({(-0.5+\lx)*2*\halfwidth+3}, 0);

    \fill (0, 0) circle ({0.12*\lx});
    \fill (\zunit, 0) circle ({0.12*\lxp});
    %  \fill[hl-a] (\zunit, 0.5) circle ({0.12*\lxpp});
    
    \kftextin[in=\ta, duration=\dt, out=\tend+1, offset fst=0.6, offset snd=3*pi/4]{0}{0}{$0$}
    %  \kftextin[in=(\ta+\dt), duration=\dt, offset fst=0.6, offset snd=-pi/4]{\zunit}{0}{$1$}

    % draw displacement & velocity
    \newcommand\xvc{1.2*\zunit}
    \kftween[in=\tc, duration=\dt]{0}{\zunit}
    \pgfmathsetmacro\xf\pgfmathresult
    \kftween[in=\td, duration=\dt]{0}{\xvc}
    \pgfmathsetmacro\xv\pgfmathresult

    \kftween[in=\te, duration=\dt]{0}{1}
    \pgfmathsetmacro\xd\pgfmathresult
    
    \pgfmathsetmacro\xfo{0.16}
    \kfclip{\tc}{\tend+1}{ %{\te+\dt}{
%      \draw[->, >=stealth] (\xfo+\xf*\xd, 0) -- (\xf-\xfo, 0);
      \draw[->, >=stealth] (\xfo, 0) -- (\xf-\xfo, 0);
    }

    \kfclip{\td}{\te+\dt}{
      \draw[->, >=stealth] (\zunit, 0) -- ({\xvc+(1-\xd)*\xv}, 0);
    }

    \kftextin[in=\tc, duration=\dt, out=(\tc+\tdelay), offset fst=0.7, offset snd=-pi/2]{0.5*\zunit}{0}{\scriptsize displacement}
    \kftexttween[in=(\tc+\tdelay), duration=\dt, out=\ts, offset fst=0.7, offset snd=-pi/2]{0.5*\zunit}{0}{\scriptsize displacement}{0.5*\zunit}{0}{$f$}
    \kftexttween[in=\ts, duration=\dt, out=\tend+1, offset fst=0.7, offset snd=-pi/2]{0.5*\zunit}{0}{$f$}{0.5*\zunit}{0.05}{$e^{ct}$}


    % \kftexttween[in=(\tc+\tdelay), duration=\dt, out=\te, offset fst=0.8, offset snd=-pi/2]{0.5*\zunit}{0}{\scriptsize displacement}{0.5*\zunit}{0}{$f$}
    %\kftextout[in=\te, out=(\te+\dt), offset fst=0.6, offset snd=pi/2]{0.5*\zunit}{0}{$f$}

    \kftextin[in=\td, duration=\dt, out=(\td+\tdelay), offset fst=0.8, offset snd=pi/2]{\zunit+\xvc/2}{0}{\scriptsize velocity}
    \kftexttween[in=(\td+\tdelay), duration=\dt, out=\ts+\dt,  offset fst=0.8, offset snd=pi/2]{\zunit+\xvc/2}{0}{\scriptsize velocity}{\zunit+\xvc/2}{0}{$\frac{df}{dt}$}
    \kftexttween[in=\ts+\dt, duration=\dt, out=\te+\dt,  offset fst=0.8, offset snd=pi/2]{\zunit+\xvc/2}{0}{$\frac{df}{dt}$}{\zunit+\xvc/2}{0}{$\phantom{e^{ct}}\frac{d}{dt}e^{ct}$}

\kftextout[in=\te+\dt, out=(\te+2*\dt), offset fst=0.8, offset snd=pi/2]{\zunit+\xvc/2}{0}{$\phantom{e^{ct}}\frac{d}{dt}e^{ct}$}


    
  \end{scope}
}

% \kfframes{8}{

  \newcommand\ta{0}
  \newcommand\taa{1}
  \newcommand\tb{2} % lift arrow
  \newcommand\tc{4}
  \newcommand\td{6}

  \kftween[in=\ta, duration=\dt]{\halfwidth}{\halfwidth-1.6}
  \pgfmathsetmacro\xh\pgfmathresult

  % phantom spaghetti
  \kftext[polar=false]{\xh}{6}{$\begin{aligned}\phantom{\frac{df}{dt}} &\phantom{=} \;\:c \phantom{f}\\\phantom{f(0) }&\phantom{= 1}\end{aligned}$}
  \kftextin[in=\taa, out=\tend+1, duration=\dt, polar=false]{\halfwidth}{6.5}{$\phantom{c} > 0$}

  \begin{scope}[shift={({\halfwidth-3}, {\halfheight-2})}]
    \pgfmathsetmacro\zunit{3}

    % axis & leftovers
    \draw[bg-b] (-\halfwidth+3, 0) -- ({\halfwidth+3}, 0);

    \fill (0, 0) circle (0.12);
    \fill (\zunit, 0) circle (0.12);

    \kftext[offset fst=0.6, offset snd=3*pi/4]{0}{0}{$0$}
    
    \pgfmathsetmacro\xfo{0.16}
    \draw[->, >=stealth] (\xfo, 0) -- (\zunit-\xfo, 0);    
    \kftext[offset fst=0.7, offset snd=-pi/2]{0.5*\zunit}{0.05}{$e^{ct}$}

    % actual animation
    \kfclip{\tb}{\tend+1}{
      \pgfmathsetmacro\dxl{\zunit/4}
      
      \kftween[in=\tb, duration=\dt]{0}{\zunit/2}
      \pgfmathsetmacro\xax\pgfmathresult
      \kftween[in=\td, duration=\dt]{0}{\zunit/2+\dxl}
      \pgfmathsetmacro\xax{\xax+\pgfmathresult}
      
      \kftween[in=\tb, duration=\dt]{0}{2}
      \pgfmathsetmacro\xay\pgfmathresult
      \kftween[in=\td, duration=\dt]{0}{-2}
      \pgfmathsetmacro\xay{\xay+\pgfmathresult}
      
      \kftween[in=\tc, duration=\dt]{0}{\dxl}
      \pgfmathsetmacro\dxh\pgfmathresult
      
      
      \draw[->, >=stealth] (\xax+\xfo-\dxh, \xay) -- (\xax+\zunit+\dxh-\xfo, \xay);
      \kftextin[in=\tb+\dt, duration=\dt, out=\tc, polar=false]{\zunit/2+\xax}{-0.6+\xay}{$e^{ct}$}
      \kftextout[in=\tc, out=\tc+\dt, polar=false]{\zunit/2+\xax}{-0.6+\xay}{$e^{ct}$}
      
      \kftextin[in=\tc+\dt, out=\td, duration=\dt, polar=false]{\zunit/2+\xax}{0.6+\xay}{$c e^{ct}$}
      \kftextout[in=\td, duration=\dt, out=\td+2*\dt, polar=false]{\zunit/2+\xax}{0.6+\xay}{$c e^{c}$}
      \kftextin[in=\td+\dt, out=\tend+1, duration=\dt, polar=false]{\zunit/2+\xax}{0.8+\xay}{$\frac{d}{dt}e^{ct}$}
    }

    
  \end{scope}
}

% \kfframes{6}{

  \newcommand\taa{0} % scale
  \newcommand\ta{0} % scale& show t counter
  \newcommand\tb{1} % particle moves
  \newcommand\tc{4} % clip
  \newcommand\td{5} % fadeout
  
  % phantom spaghetti
  \kftext[polar=false]{\halfwidth-1.6}{6}{$\begin{aligned}\phantom{\frac{df}{dt}} &\phantom{=} \;\:c \phantom{f}\\\phantom{f(0) }&\phantom{= 1}\end{aligned}$}
  \kftext[polar=false]{\halfwidth}{6.5}{$\phantom{c} > 0$}

  \pgfmathsetmacro\tt{max(0, \t-\tb-\dt)}
  \kftextin[in=\ta, duration=\dt, out=\td, polar=false]{\halfwidth}{5.5}{$t = \pgfmathprintnumber[fixed,precision=3, fixed zerofill=true]{\tt}$}
  \kftextout[in=\td, out=\td+\dt, polar=false]{\halfwidth}{5.5}{$t = \pgfmathprintnumber[fixed,precision=3, fixed zerofill=true]{\tt}$}


  \kftween[in=\taa, duration=\dt]{3}{6}
  \pgfmathsetmacro\ssx\pgfmathresult
  \begin{scope}[shift={({\halfwidth-\ssx}, {\halfheight-2})}]

    % zoom out yo
    \kftween[in=\taa, duration=\dt]{3}{1}
    \pgfmathsetmacro\zunit{\pgfmathresult}

    % axis & leftovers
    \draw[bg-b] (-\halfwidth+\ssx, 0) -- ({\halfwidth+\ssx}, 0);

    \pgfmathsetmacro\xfo{0.16}

    % fade out the line too
    \fill[bg-b] (\zunit, 0) circle (0.12);

    \kfclip{\tc}{\tend+1}{
      \kftween[in=\td, duration=\dt]{100}{0}
      \pgfmathsetmacro\cxc\pgfmathresult
      \draw[fg-a!\cxc!bg-b] (0, 0) -- ({\halfwidth+\ssx}, 0);
    }

    \fill (0, 0) circle (0.12);

    \kftext[offset fst=0.6, offset snd=3*pi/4]{0}{0}{$0$}
    \kftextin[in=\ta, duration=\dt, out=\tend+1, offset fst=0.6, offset snd=3*pi/4]{\zunit}{0}{$1$}
    
    % position & velocity vectors & marker
    \kfclip{0}{\tc}{
      \pgfmathsetmacro\xx{\zunit*e^(1.8*\tt)}

      \fill (\xx, 0) circle (0.12);
      
      \draw[->, >=stealth] (\xfo, 0) -- (\xx-\xfo, 0);    
      \kftext[offset fst=0.7, offset snd=-pi/2]{0.5*\xx}{0}{$e^{ct}$}

      \pgfmathsetmacro\dxl{\zunit/4}
      \pgfmathsetmacro\xas{(2*\dxl+\zunit)/\zunit}

      \kftext[polar=false]{\xx+\xx*\xas/2}{0.8}{$\frac{d}{dt}e^{ct}$}
      \draw[->, >=stealth] (\xx, 0) -- ({\xx*(1+\xas)-\xfo}, 0);
    }


  \end{scope}
}

% \kfframes{3}{

  \newcommand\ta{0} % restore axes
  \newcommand\tb{2} % change sign 
  \newcommand\tc{0} % draw f
  
  % phantom spaghetti
  \kftext[polar=false]{\halfwidth-1.6}{6}{$\begin{aligned}\phantom{\frac{df}{dt}} &\phantom{=} \;\:c \phantom{f}\\\phantom{f(0) }&\phantom{= 1}\end{aligned}$}
  \kftext[polar=false]{\halfwidth}{6.5}{$\phantom{c >}\;0$}
  \kftextout[in=0, duration=\dt, out=\tb+\dt, polar=false]{\halfwidth}{6.5}{$\phantom{c} > \phantom{0}$}
  \kftextin[in=\tb, duration=\dt, out=\tend+1, polar=false]{\halfwidth}{6.5}{$\phantom{c} < \phantom{0}$}

  \begin{scope}[shift={({\halfwidth-6}, {\halfheight-2})}]

    \kftween[in=\ta, duration=\dt]{1}{12}
    \pgfmathsetmacro\zunit{\pgfmathresult}

    % axis & leftovers
    \draw[bg-b] (-\halfwidth+6, 0) -- ({\halfwidth+6}, 0);

    \pgfmathsetmacro\xfo{0.16}
    
    \fill (0, 0) circle (0.12);
    \fill[bg-b] (\zunit, 0) circle (0.12);

    \kftext[offset fst=0.6, offset snd=3*pi/4]{0}{0}{$0$}

    \kftween[in=\ta, duration=\dt]{3*pi/4}{pi/4}
    \pgfmathsetmacro\lxth\pgfmathresult
    \kftext[offset fst=0.6, offset snd=\lxth]{\zunit}{0}{$1$}

    % f
    \kftween[in=\tc, duration=\dt]{0}{0.12}
    \pgfmathsetmacro\xd\pgfmathresult
    
    \kftween[in=\tc+\dt, duration=\dt]{0}{\zunit}
    \pgfmathsetmacro\xl\pgfmathresult
    
    \fill (\zunit, 0) circle (\xd);
    \draw[->,>=stealth] (\xfo, 0) -- (\xl-\xfo, 0);

    \kftextin[in=\tc+\dt, duration=\dt, polar=false]{\zunit/2}{-0.6}{$e^{ct}$}
    
  \end{scope}
}

% \kfframes{5}{

  \newcommand\ta{0} % lift arrow
  \newcommand\tb{2} % scale arrow
  \newcommand\tc{4} % drop arrow
  
  % phantom spaghetti
  \kftext[polar=false]{\halfwidth-1.6}{6}{$\begin{aligned}\phantom{\frac{df}{dt}} &\phantom{=} \;\:c \phantom{f}\\\phantom{f(0) }&\phantom{= 1}\end{aligned}$}
  \kftext[polar=false]{\halfwidth}{6.5}{$\phantom{c} < 0$}

  \begin{scope}[shift={({\halfwidth-6}, {\halfheight-2})}]

    \pgfmathsetmacro\zunit{12}

    % axis & leftovers
    \draw[bg-b] (-\halfwidth+6, 0) -- ({\halfwidth+6}, 0);

    \pgfmathsetmacro\xfo{0.16}
    
    \fill (0, 0) circle (0.12);
    \fill[bg-b] (\zunit, 0) circle (0.12);

    \kftext[offset fst=0.6, offset snd=3*pi/4]{0}{0}{$0$}

    \kftext[offset fst=0.6, offset snd=pi/4]{\zunit}{0}{$1$}

    \fill (\zunit, 0) circle (0.12);
    \draw[->,>=stealth] (\xfo, 0) -- (\zunit-\xfo, 0);

    \kftext[polar=false]{\zunit/2}{-0.6}{$f$}

    % actual animation. gosh, i hope i never need to come back and
    % change this because this will be illegible within two days.

    \kftween[in=\ta, duration=\dt]{0}{2}
    \pgfmathsetmacro\xay\pgfmathresult
    \kftween[in=\tc, duration=\dt]{0}{-2}
    \pgfmathsetmacro\xay{\xay+\pgfmathresult}

    \kftween[in=\tc, duration=\dt]{1}{0}
    \pgfmathsetmacro\xag\pgfmathresult
    
    \kftween[in=\tb, duration=\dt]{1}{-0.4}
    \pgfmathsetmacro\xas\pgfmathresult

    \kftween[in=\tc, duration=\dt]{0}{(\zunit*(1-abs(\xas)))/2}
    \pgfmathsetmacro\xax\pgfmathresult
    
    \kfclip{\ta}{\tend+1}{
      \pgfmathsetmacro\xll{\xax+\zunit*(0.5-\xas/2)+\xfo*\xag}
      \pgfmathsetmacro\xlr{\xax+\zunit*(0.5+\xas/2)-\xfo*\xag}

      % duplicate the dot so it doesn't get covered.
      \kfclip{\tc}{\tend+1}{
        \draw[line width=1mm, line cap=round, bg-a] (\xll, +\xay) -- (\xlr, +\xay);
        \fill (\zunit, 0) circle (0.12);
      }
      
      \draw[->,>=stealth] (\xll, +\xay) -- (\xlr, +\xay);
      \kftextin[in=\ta+\dt, duration=\dt, out=\tb, polar=false]{\zunit/2}{\xay-0.6}{$f$}
      \kftextout[in=\tb, duration=\dt, out=\tb+\dt, polar=false]{\zunit/2}{\xay-0.6}{$f$}

      \kftextin[in=\tb+\dt, duration=\dt, out=\tend+1, polar=false]{\xax+\zunit/2}{\xay+0.7}{$\frac{df}{dt}$}
      
    }
       
  \end{scope}
}

% \kfframes{8}{

  \newcommand\ta{0} % show timer
  \newcommand\tb{1} % start moving
  \newcommand\tc{7} % fade out
  
  % phantom spaghetti
  \kftext[polar=false]{\halfwidth-1.6}{6}{$\begin{aligned}\phantom{\frac{df}{dt}} &\phantom{=} \;\:c \phantom{f}\\\phantom{f(0) }&\phantom{= 1}\end{aligned}$}
  \kftext[polar=false]{\halfwidth}{6.5}{$\phantom{c} < 0$}


  \pgfmathsetmacro\tt{max(0, \t-\tb-\dt)}
  \kftextin[in=\ta, duration=\dt, out=\tc, polar=false]{\halfwidth}{5.5}{$t = \pgfmathprintnumber[fixed,precision=3, fixed zerofill=true]{\tt}$}
  \kftextout[in=\tc, out=\tc+\dt, polar=false]{\halfwidth}{5.5}{$t = \pgfmathprintnumber[fixed,precision=3, fixed zerofill=true]{\tt}$}


  
  \begin{scope}[shift={({\halfwidth-6}, {\halfheight-2})}]

    \pgfmathsetmacro\zunit{12}

    % axis & leftovers
    \draw[bg-b] (-\halfwidth+6, 0) -- ({\halfwidth+6}, 0);

    \pgfmathsetmacro\xfo{0.16}
    
    \fill[bg-b] (\zunit, 0) circle (0.12);

    \kftext[offset fst=0.6, offset snd=3*pi/4]{0}{0}{$0$}

    \kftext[offset fst=0.6, offset snd=pi/4]{\zunit}{0}{$1$}

    \pgfmathsetmacro\xx{\zunit*e^(-\tt)}
    \pgfmathsetmacro\xas{-0.4}

    % displacement arrow
    \draw (\xfo, 0) -- (\xx-\xfo, 0);

    % velocity arrow
    \draw[line width=1mm, line cap=round, bg-a] (\xx, 0) -- (\xx+\xx*\xas, 0);
    \kfclip{0}{\tb+4}{      
      \draw[->,>=stealth] (\xx, 0) -- (\xx+\xx*\xas, 0);
    }
    \kftween[in=\tc, duration=\dt]{0.12}{0}
    \pgfmathsetmacro\xd\pgfmathresult
    \fill (\xx, 0) circle (\xd);
    
    \kftextout[duration=\dt, out=\tc+\dt, polar=false]{\xx/2}{-0.6}{$f$}
    \kftextout[duration=\dt, out=\tc+\dt, polar=false]{\xx+\xx*\xas/2}{+0.7}{$\frac{df}{dt}$}

    % zero on top layer
    \fill (0, 0) circle (0.12);
    
  \end{scope}
}

% \kfframes{16}{

\newcommand\ta{0} % rescale axes
\newcommand\tb{2} % imaginary!
\newcommand\tc{0} % fill f

\newcommand\thint{3} % hint e^it
\newcommand\tdelay{2}

\newcommand\td{9} % copy f
\newcommand\te{11} % arg in
\newcommand\tf{12.5} % arg out
\newcommand\tg{14} % df/dt


% phantom spaghetti
\kftext[polar=false]{\halfwidth-1.6}{6}{$\begin{aligned}\phantom{\frac{df}{dt}} &\phantom{=} \;\:c \phantom{f}\\\phantom{f(0) }&\phantom{= 1}\end{aligned}$}
\kftextout[in=0, out=\tb+\dt, polar=false]{\halfwidth}{6.5}{$\phantom{c} < 0$}
\kftextin[in=\tb, duration=\dt, out=\tend+1, polar=false]{\halfwidth}{6.5}{$\phantom{c} = i$}

%  \kftextin[in=\tb+\dt, duration=\dt, polar=false]{\halfwidth+2}{6.4}{$(?)$}


\kftween[in=\ta, duration=\dt]{6}{0}
\pgfmathsetmacro\sxh\pgfmathresult
\kftween[in=\ta, duration=\dt]{pi/4}{-pi/4}
\pgfmathsetmacro\ola\pgfmathresult


\begin{scope}[shift={({\halfwidth-\sxh}, {\halfheight-2})}]

\kftween[in=\ta, duration=\dt]{12}{3}
\pgfmathsetmacro\zunit\pgfmathresult

% axis & leftovers
\draw[bg-b] (-\halfwidth+\sxh, 0) -- ({\halfwidth+\sxh}, 0);


\fill[bg-b] (\zunit, 0) circle (0.12);
\fill (0, 0) circle (0.12);

\kftext[offset fst=0.6, offset snd=3*pi/4]{0}{0}{$0$}
\kftext[offset fst=0.6, offset snd=\ola]{\zunit}{0}{$1$}

% displacement arrow
\kftween[in=\tc+\dt, duration=\dt]{0}{\zunit}
\pgfmathsetmacro\xl\pgfmathresult

\kftween[in=\tc, duration=\dt]{0}{0.12}
\pgfmathsetmacro\xd\pgfmathresult

\pgfmathsetmacro\xfo{0.16}

\kfclip{\tc+\dt}{\tend+1}{
\draw[->, >=stealth] (\xfo, 0) -- (\xl-\xfo, 0);
\fill (\zunit, 0) circle (\xd);
}

\kftextin[in=\tc+\dt, duration=\dt, out=\thint, polar=false]{\zunit/2}{-0.6}{$f$}
\kftextout[in=\thint, duration=\dt, out=\thint+\dt, polar=false]{\zunit/2}{-0.6}{$f$}
\kftextin[in=\thint+\tdelay, duration=\dt, out=\tend+1, polar=false]{\zunit/2}{-0.6}{$f$}

\kftextin[in=\thint, out=\thint+\tdelay, duration=\dt, polar=false]{\zunit/2}{-1}{$f(t) \stackrel{?}{=} e^{it}$}
\kftextout[in=\thint+\tdelay, duration=\dt, out=(\thint+\tdelay+\dt), polar=false]{\zunit/2}{-1}{$f(t) \stackrel{?}{=} e^{it}$}


% copy of displacement arrow
\kfclip{\td}{\tend+1}{
\kftween[in=\td, duration=\dt]{pi}{0}
\pgfmathsetmacro\fadth\pgfmathresult
\pgfmathsetmacro\fadx{\zunit*cos(\fadth r)/2}
\pgfmathsetmacro\fady{\zunit*sin(\fadth r)/2}

\kftween[in=\te, duration=\dt]{0}{pi/2}
\pgfmathsetmacro\faeth\pgfmathresult
\pgfmathsetmacro\faex{\zunit*cos(\faeth r)}
\pgfmathsetmacro\faey{\zunit*sin(\faeth r)}

\draw[->, >=stealth] (\zunit/2+\fadx, \fady) -- (\zunit/2+\fadx+\faex, \fady+\faey);

\kftextin[in=\td+\dt, duration=\dt, out=\te-\dt, polar=false]{\zunit*1.5}{-0.6}{$f$}
\kftextout[in=\te-\dt, duration=\dt, out=\te, polar=false]{\zunit*1.5}{-0.6}{$f$}

% stealth
\begin{scope}[shift={(\zunit, 0)}]
\zargin[in=\te, out=\tf]{pi/2}
\zargout[in=\tf, out=\tf+\dt]{pi/2}
\kftextin[in=\te, duration=\dt, out=\tf]{1.6}{pi/5}{$\frac{\pi}{2}$}
\kftextout[in=\tf, out=\tf+\dt]{1.6}{pi/5}{$\frac{\pi}{2}$}

\kftextin[in=\tf+\dt, duration=\dt, out=\tg, polar=false]{-0.6}{\zunit/2}{$if$}
\kftextout[in=\tg, out=\tg+\dt, polar=false]{-0.6}{\zunit/2}{$if$}

\kftextin[in=\tg+\dt, duration=\dt, out=\tend+1, polar=false]{0.6}{\zunit/2}{$\frac{df}{dt}$}


\end{scope}

}

\end{scope}
}

% \kfframes{4}{

  \newcommand\ta{2}
  
  % phantom spaghetti
  \kftextout[in=0, duration=\dt, out=\ta+\dt, polar=false]{\halfwidth-1.6}{6}{$\begin{aligned}\phantom{\frac{df}{dt}} &\phantom{=} \;\:c \phantom{f}\\\phantom{f(0) }&\phantom{= 1}\end{aligned}$}
  \kftextout[in=0, duration=\dt, out=\ta+\dt, polar=false]{\halfwidth}{6.5}{$\phantom{c} = i$}
  
  \begin{scope}[shift={(\halfwidth, {\halfheight-2})}]

    \pgfmathsetmacro\zunit{3}

    % axis & leftovers
    \draw[bg-b] (-\halfwidth, 0) -- (\halfwidth, 0);

    % grow y axis
    \kftween[in=\ta, duration=\dt]{0}{1}
    \pgfmathsetmacro\xys\pgfmathresult
    \pgfmathsetmacro\xyc{100*\xys}

    \draw[bg-b] (0, {-\xys*(\halfheight+2)}) -- (0, {\xys*(\halfheight+2)});
    
    \fill[bg-b] (\zunit, 0) circle (0.12);
    \fill[bg-b!\xyc!fg-a] (0, 0) circle ({0.12*(1-\xys)});
    
    \kftextout[duration=\dt, out=\ta+\dt, offset fst=0.6, offset snd=3*pi/4]{0}{0}{$0$}
    \kftext[offset fst=0.6, offset snd=-pi/4]{\zunit}{0}{$1$}

    \pgfmathsetmacro\xfo{0.16}

    \draw[->, >=stealth] ({\xfo*(1-\xys)}, 0) -- (\zunit-\xfo, 0);
    \fill (\zunit, 0) circle (0.12);

    \kftext[polar=false]{\zunit/2}{-0.6}{$f$}

    \begin{scope}[shift={(\zunit, 0)}]
    \draw[->, >=stealth] (0, 0) -- (0, \zunit);
    \kftext[polar=false]{0.6}{\zunit/2}{$\frac{df}{dt}$}
    \end{scope}
    
  \end{scope}
}


%% 0.5 RETURN OF THE EXPONENTIAL
% \kfframes{4}{\ztitle{The Return of the Exponential}}
% \kfframes{15}{
  \pgfmathsetmacro\za{1}
  \pgfmathsetmacro\zb{1.2}
  
  \newcommand\ta{2} % begin first rotation
  \newcommand\tdelay{1.5}
  \newcommand\tb{5}
%  \newcommand\tc{8} % a-followed-by-b end
  \newcommand\td{9} % a + b start - 1 appears
  \newcommand\te{10} % a + b rotation start
  \newcommand\tf{14} % a + b z out
  \newcommand\tg{15} % grid out

  
  
  \kftween[in=\ta, duration=\dt]{0}{\za}
  \pgfmathsetmacro\zat\pgfmathresult
  \kftween[in=\tb, duration=\dt]{0}{\zb}
  \pgfmathsetmacro\zad{\pgfmathresult}
  \pgfmathsetmacro\zat{\zat+\zad}
  
  
  \zgrid[origin y=2, unit=3, out=\tg]{

    % for beta label
    \kftextin[in=\tb, duration=\dt, out=(\tb+\tdelay+\dt)]{1.6}{(\za+\zb/2)}{\colorbox{bg-a}{\phantom{$\beta$}}}
    \kftextout[in=(\tb+\tdelay), duration=\dt, out=(\tb+\tdelay+\dt)]{1.6}{(\za+\zb/2)}{\colorbox{bg-a}{\phantom{$\beta$}}}
    \kftextin[in=(\te+\tdelay-\dt), duration=\dt, out=\tend, offset fst=0.6, offset snd=pi/3]
             {\zunit}{\zat}{\colorbox{bg-a}{\phantom{$e^{i\alpha} \cdot e^{i\beta} = e^{i(\alpha + \beta)}$}}}

    % ghost circle
    \kfclip{(\tb+\tdelay)}{\te+\tdelay+\dt}{
      \fill[bg-b] ({\zunit*cos(\zat r)}, {\zunit*sin(\zat r)}) circle (0.12);
    }

    % stuff.
    \zin[arg visible=false, out=\tb+\tdelay]{1}{\zat}

    \kftextin[duration=\dt, out=\ta-\dt, offset fst=0.6, offset snd=-pi/4]{\zunit}{0}{$1$}
    \kftextout[in=(\ta-\dt), out=\ta, offset fst=0.6, offset snd=-pi/4]{\zunit}{0}{$1$}

    \zargout[out=\ta+\tdelay+\dt]{\zat}
    \kftextin[in=\ta, duration=\dt, out=(\ta+\tdelay)]{1.6}{\za/2}{$\alpha$}
    \kftextout[in=(\ta+\tdelay), duration=\dt, out=(\ta+\tdelay+\dt)]{1.6}{\za/2}{$\alpha$}

    \kftextin[in=(\ta+\tdelay), duration=\dt, out=(\tb-\dt), offset fst=0.8, offset snd=pi/8]{\zunit}{\zat}{$e^{i\alpha}$}
    \kftextout[in=(\tb-\dt), duration=\dt, out=\tb,      offset fst=0.8, offset snd=pi/8]{\zunit}{\zat}{$e^{i\alpha}$}

    \zargout[in=\tb, out=(\tb+\tdelay), arg origin=\za]{\zat}

    \kftextin[in=\tb, duration=\dt, out=(\tb+\tdelay)]{1.6}{(\za+\zb/2)}{$\beta$}
    \kftextout[in=(\tb+\tdelay), duration=\dt, out=(\tb+\tdelay+\dt)]{1.6}{(\za+\zb/2)}{$\beta$}

    \kftextin[in=(\tb+\tdelay), duration=\dt, out=\te+\tdelay, offset fst=0.6, offset snd=pi/3]
             {\zunit}{\zat}{$e^{i\alpha} \cdot e^{i\beta} \;\phantom{ = e^{i(\alpha + \beta)}}$}

    \zout[in=\tb, out=\tb+\tdelay+\dt, arg visible=false]{1}{\zat}


    % SECOND BIT.
    \zin[in=\td, out=\te]{1}{0}
    \kftextin[in=\td, duration=\dt, out=\te-\dt, offset fst=0.6, offset snd=-pi/4]{\zunit}{0}{$1$}
    \kftextout[in=(\te-\dt), out=\te, offset fst=0.6, offset snd=-pi/4]{\zunit}{0}{$1$}

    \kftween[in=\te, duration=\dt]{0}{(\za+\zb)}
    \pgfmathsetmacro\zat\pgfmathresult

    \zout[in=\te, out=(\te+\tdelay+\dt), arg visible=false]{1}{\zat}
    \zargout[in=\te, out=\te+\tdelay+\dt]{\zat}

    \kftextin[in=\te, duration=\dt, out=(\te+\tdelay)]{2}{(\za+\zb)/3}{$\alpha+\beta$}
    \kftextout[in=(\te+\tdelay), duration=\dt, out=(\te+\tdelay+\dt)]{2}{(\za+\zb)/3}{$\alpha+\beta$}

    \kftexttween[in=(\te+\tdelay), duration=\dt, out=\tend+1, offset fst=0.6, offset snd=pi/3]
             {\zunit}{\zat}{$e^{i\alpha} \cdot e^{i\beta} \;\phantom{ = e^{i(\alpha + \beta)}}$}
             {\zunit}{\zat}{$e^{i\alpha} \cdot e^{i\beta} = e^{i(\alpha + \beta)}$}

    
    %% \zargout[in=\te, arg visible=false, out=\te+\tdelay]{1}{\zat}
    
  }
}

% \kfframes{6}{
  \newcommand\ta{0}
  \newcommand\tb{1} % fade in sin/cos
  \newcommand\tc{2} % move row down
  \newcommand\td{2.5} % expand
  

  \pgfmathsetmacro\tx{\halfwidth+3*cos(2.2 r)+0.6*cos(pi/3 r)}
  \pgfmathsetmacro\ty{2+3*sin(2.2 r)+0.6*sin(pi/3 r)}
  
  \kftween[in=\ta, duration=\dt]{\ty}{6.1}
  \pgfmathsetmacro\ya\pgfmathresult

  \kftween[in=\tb, duration=\dt]{0}{(4-\tx)}
  \pgfmathsetmacro\xb\pgfmathresult

  \kftween[in=\tc-\dt, duration=\dt]{4.8}{3.8}
  \pgfmathsetmacro\yb\pgfmathresult
  
  \kftextout[duration=\dt, out=\tb, polar=false]{\tx}{\ya}{$e^{i\alpha} \cdot e^{i\beta}\;\phantom{ = e^{i(\alpha + \beta)}}$}
  \kftextout[duration=\dt, out=(\tb+\dt), polar=false]{(\xb+\tx)}{\ty}{$\phantom{e^{i\alpha} \cdot e^{i\beta}}= e^{i(\alpha + \beta)}$}

  \kftextin[duration=\dt, in=(\tb-\dt), out=\td, polar=false]{9}{6}{$(\cos \alpha + i \sin \alpha) \cdot (\cos \beta + i \sin \beta)$}
  \kftextout[in=\td, duration=\dt, out=(\td+\dt), polar=false]{9}{6}{$(\cos \alpha + i \sin \alpha) \cdot (\cos \beta + i \sin \beta)$}

  \kftextin[duration=\dt, in=\tb, polar=false]{(4.2+\tx+\xb)}{\yb}{$= \cos (\alpha + \beta) + i \sin (\alpha + \beta)$}

  \kftextin[duration=\dt, in=\td, polar=false]{7.84}{6}{$(\cos \alpha \cos \beta - \sin \alpha \sin \beta)$}
  \kftextin[duration=\dt, in=\td, polar=false]{9}{5}{$+\;i\:(\cos \alpha \sin \beta + \sin \alpha \cos \beta)$}

}

% % fix textcolor.
\makeatletter
\renewcommand*{\@textcolor}[3]{%
  \protect\leavevmode
  \begingroup
    \color#1{#2}#3%
  \endgroup
}
\makeatother

\kfframes{13}{
  \newcommand\ta{0} % highlight cos
  \newcommand\tb{6} % highlight sin

  \newcommand\tc{12} % fade out

  
  \kftween[in=\ta, duration=\dt]{100}{0}
  \colorlet{xc}{fg-a!\pgfmathresult!bg-b}

  \kftween[in=\tb, duration=\dt]{100}{0}
  \colorlet{xcc}{fg-a!\pgfmathresult!bg-b}

  \kftween[in=\tb, duration=\dt]{100}{0}
  \colorlet{xcs}{xc!\pgfmathresult!fg-a}

  \kftween[in=\tc, duration=\dt]{100}{0}
  \colorlet{xcc}{xcc!\pgfmathresult!bg-b}
  \colorlet{xcs}{xcs!\pgfmathresult!bg-b}

  % fadeout
  \kftween[in=\tc+\dt, duration=\dt]{100}{0}
  \colorlet{xcc}{xcc!\pgfmathresult!bg-a}
  \colorlet{xcs}{xcs!\pgfmathresult!bg-a}
  \colorlet{xc}{xc!\pgfmathresult!bg-a}  
  
    \kftext[polar=false]{7.84}{6}{$\textcolor{xc}{(}\textcolor{xcc}{\cos \alpha \cos \beta - \sin \alpha \sin \beta}\textcolor{xc}{)}$}
    \kftext[polar=false]{9}{5}{$\textcolor{xc}{+\;i\:(}\textcolor{xcs}{\cos \alpha \sin \beta + \sin \alpha \cos \beta}\textcolor{xc}{)}$}

    \kftext[polar=false]{8.2}{3.8}{$\textcolor{xc}{=}\textcolor{xcc}{\cos (\alpha + \beta)}\textcolor{xc}{+ i }\textcolor{xcs}{\sin (\alpha + \beta)}$}

}

% \kfframes{16}{
  \newcommand\ta{1} % 1 in
  \newcommand\tb{3} % rotate by n-theta
  \newcommand\tc{4} % n-theta out

  \newcommand\td{5}
  \newcommand\tdelay{1.5}
  
  \newcommand\tf{15}

  \pgfmathsetmacro\th{0.75}
  \pgfmathsetmacro\n{3}
  \pgfmathsetmacro\m{\n-1}

  \pgfmathsetmacro\te{\td+(\n+1)*\tdelay}

  
  \zgrid[in=0, origin y=2.5, unit=3]{

    % backgroudn stuff
    \kftextin[in=(\te), duration=\dt, offset fst=1.5, offset snd=0.4]{\zunit}{\n*\th}{\colorbox{bg-a}{\phantom{$e^{in\theta} = (e^{i\theta})^n$}}}
    
    \kfclip{\tc}{\tend}{
      \fill[bg-b] ({\zunit*cos(\n*\th r)}, {\zunit*sin(\n*\th r)}) circle (0.12);
    }
    
    \zin[in=\ta, out=\tb]{1}{0}
    \kftextin[in=\ta, duration=\dt, out=\tb, offset fst=0.6, offset snd=-pi/4]{\zunit}{0}{$1$}
    \kftextout[in=\tb, out=\tb+\dt, offset fst=0.6, offset snd=-pi/4]{\zunit}{0}{$1$}
    

    \kftextin[in=\tb, duration=\dt, out=\tc]{1.7}{pi/4}{$n\theta$}
    \kftextout[in=\tc, out=\tc+\dt]{1.7}{pi/4}{$n\theta$}

    \ztween[in=\tb, out=\tc, stagger=false]{1}{0}{1}{\n*\th}

    \zout[in=\tc, out=\tc+\dt]{1}{\n*\th}

    \kftextin[in=\tc, duration=\dt, offset fst=1.5, offset snd=0.4]{\zunit}{\n*\th}{$e^{in\theta} \phantom{ = (e^{i\theta})^n}$}


    % n-times-theta
    \zin[in=\td, out=(\td+\tdelay)]{1}{0}
    
    \foreach \j in {1, ..., \n}{
      \pgfmathsetmacro\ta{\td+\j*\tdelay}     % beginning of this
      \pgfmathsetmacro\tb{\td+(\j+1)*\tdelay} % beginning of next

      \ztween[in=\ta, out=\tb, stagger=false, arg visible=false]{1}{\th*(\j-1)}{1}{\th*\j}
      
      \kftextin[in=\ta, duration=\dt, out=(\tb-\dt)]{1.5}{\th*(\j-0.5)}{$\theta$}
      \kftextout[in=(\tb-\dt), duration=\dt, out=\tb]{1.5}{\th*(\j-0.5)}{$\theta$}
      
      \zargin[in=\ta, out=\tb-\dt, arg origin=(\th*(\j-1))]{\th*\j}
      \zargout[in=\tb-\dt, out=\tb, arg origin=(\th*(\j-1))]{\th*\j}

    }

    \z[in=\te, arg visible=false]{1}{\n*\th}
    \kftextin[in=(\te+\dt), duration=\dt, offset fst=1.5, offset snd=0.4]{\zunit}{\n*\th}{$\phantom{e^{in\theta}} = (e^{i\theta})^n$}
    
  }
}



\end{document}
